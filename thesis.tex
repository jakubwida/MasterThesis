\documentclass[12pt, oneside]{report}

\usepackage[left=2.5cm, top=2.5cm, bottom=2.5cm, right=2.5cm]{geometry}

\usepackage[utf8]{inputenc}
\usepackage[T1]{polski}
\usepackage[polish,english]{babel}
\selectlanguage{polish}

\begin{document}
\thispagestyle{empty}
\begin{titlepage}
    \begin{center}

           \Large
	\textbf{Uniwersytet Jagielloński w Krakowie}\vspace{0.2cm}\\ Wydział Fizyki, Astronomii i Informatyki Stosowanej
               \vspace*{1cm}

         \vspace{3cm}
         \Large
          \textbf{Jakub Wida}\\\vspace{0.5cm}
         \normalsize Nr albumu: 1113470\\
             \vspace{2cm}
        \Huge
        \textbf{Losowe upakowania układów złożonych z dysków z wykorzystaniem kart graficznych}

        \vspace{1.5cm}
        \normalsize
        Praca magisterska\\
        na kierunku Informatyka Stosowana\\ \vspace{0.15cm}

        \vfill
        \vspace{2cm}
       \begin{minipage}{1\textwidth}
\begin{flushright}
Praca wykonana pod kierunkiem\\
dr. hab. Michała Cieśli\\
z Zakładu Fizyki Statystycznej
\end{flushright}
\end{minipage}

        \vspace{2cm}
        \begin{center}
      Kraków 2019
        \end{center}
    \end{center}
\end{titlepage}

\newpage
 \thispagestyle{empty}
\vspace{2.5cm}
\begin{flushleft}
\large \textbf{Oświadczenie autora pracy}\vspace{0.6cm}\\
\end{flushleft}

\noindent Świadom odpowiedzialności prawnej oświadczam, że niniejsza praca dyplomowa została napisana przeze mnie samodzielnie i nie zawiera treści uzyskanych w sposób niezgodny z obowiązującymi przepisami.\\

\noindent Oświadczam również, że przedstawiona praca nie była wcześniej przedmiotem procedur związanych z uzyskaniem tytułu zawodowego w wyższej uczelni.
\vspace{2cm}
\begin{center}
\begin{tabular}{lr}
................................~~~~~~~~~~~~~~~~~~~~~~~~~~~~~~~~~~~~~~&
.......................................... \\
{~~~~Kraków, dnia} & {Podpis autora pracy~~~~}
\end{tabular}
\end{center}
\vspace{5cm}
\begin{flushleft}
\large \textbf{Oświadczenie kierującego pracą}
\end{flushleft}

\noindent Potwierdzam, że niniejsza praca została przygotowana pod moim kierunkiem i~kwalifikuje się do przedstawienia jej w postępowaniu o nadanie tytułu zawodowego.
\vspace{2cm}
\begin{center}
\begin{tabular}{lr}
................................~~~~~~~~~~~~~~~~~~~~~~~~~~~~~~~~~~~~~~&
............................................ \\
{~~~~Kraków, dnia} & {Podpis kierującego pracą~~}
\end{tabular}
\end{center}
\vfill


%INTRODUCTION ====================================================

\selectlanguage{english}

\chapter{Introduction}
{\bf TODO make this at the end, when all else is done}


%TABLE OF CONTENTS ====================================================

\tableofcontents
\newpage

%ACTUAL THESIS: INTRO ====================================================

%2-4 pages
\chapter{Problem Overwiev}
\section {Random Sequential Adsorbtion}
\subsection {Definition}
	%write about what it is, what are the issues with it, what are the goals
\subsection {Shape Types}
\subsection {Applications}
	%refer to the papers, write about applicability of the algorithm
\subsection {Thesis Goals}
	%refer to the papers, write about applicability of the algorithm
{\bf TODO unfinished, do now}

%ACTUAL THESIS: ALGORITHM OVERWIEV ====================================================

%5-15 pages
\chapter{Proposed Algorithm}
\section {Sequential Algorithm}
	%describe basic algorithm, from dr.hab. Ciesla works
\section {Parallel Application}
	%describe changes to basic algorithm, provide complete outline of target, parallel algorithm
\subsection{Block Graph}
	%draw a block graph
\section {Implementation}
	%write in general about implementing, challenges etc
\subsection{PyCuda}
	%write about language specifics, improvements over raw CUDA
\subsection{Visualisation}
	%about pyplot - shortly
\subsection{Summary}
	%stuff in general

{\bf TODO unfinished, do in near future}

%ACTUAL THESIS: RESULTS INVESTIGATION ====================================================

%15-25 pages - due to multiple pictures
\chapter{Results Examination}
\section{Performance Evaluation}
	%describe importance of performance evaluation
	%describe the algorithm runner, and hot it works
\subsection{Parameter Influence over Performance}
\subsection{Shape Influence over Performance}
{\bf TODO unfinished, do later}

\end{document}
